\documentclass[a4paper,11pt,twocolumn]{article}
\usepackage[a4paper,left=1.5cm,right=1cm,top=2cm,bottom=2cm]{geometry}
\usepackage{setspace}
\usepackage{gensymb}
\usepackage{caption}
\usepackage{graphicx}
\usepackage{tabularx}
\usepackage{lmodern}
\usepackage{watermark} 
\usepackage{lipsum}
\usepackage{xcolor}
\usepackage{listings}
\usepackage{graphicx}
\usepackage{enumitem}
\usepackage{mathtools}
\usepackage{titlesec}
\usepackage[utf8]{inputenc}
\usepackage{fontenc}
\usepackage{harvard}
\usepackage{amsfonts}
\graphicspath{{/storage/emulated/0/Download/codes/doc/figs}}
\usepackage[colorlinks,linkcolor={black},citecolor={blue!80!black},urlcolor={blue!80!black}]{hyperref}
\thiswatermark{\centering \put(-15,-100.0){\includegraphics[scale=0.3]{figs/logo.png}} }
\title{\textbf{\textsc{VERIFICATION OF BOOLEAN IDENTITIES}}}
\author{\textbf{\textit{\teflipflopxtbf{BAGILI SALMA BEGAM (FWC22130) }}}}
\begin{document}

\date{}
\maketitle
\tableofcontents


\section{PROBLEM}
\textbf{(GATE CS-2018)}
\textbf{Q.4} Let $\oplus$ and $\odot$ denote the Exclusive OR and Exclusive NOR operations,respectively. Which one of the following is NOT CORRECT?
\begin{enumerate}[label=(\Alph*)]
	\item $ (P\oplus Q)^{\prime} = (P\odot Q)$
	\item $ (P^{\prime}\oplus Q) = (P\odot Q)$
	\item $ (P^{\prime}\oplus Q^{\prime}) = (P\oplus Q)$
	\item $ (P\oplus P^{\prime})\oplus Q = (P\odot P^{\prime})\odot Q^{\prime}$
\end{enumerate}
\bigskip

\section{COMPONENTS}
	\begin{tabularx}{0.45\textwidth} {  
  | >{\centering\arraybackslash}X  
  | >{\centering\arraybackslash}X  
  | >{\centering\arraybackslash}X | } 
\hline 
\textbf{Component} &  \textbf{Value} & \textbf{Quantity}\\ 
\hline 
Arduino & UNO & 1 \\   
\hline 
Bread board & - & 1 \\ 
\hline
Jumper wires & M-M & 8 \\ 
\hline
 LED & - & 2 \\
\hline 
Resistor & 150ohms & 2\\ 
\hline 
\end{tabularx}
\bigskip
 
 \section{INTRODUCTION}
\paragraph{}
	An "identity" is merely a relationship that is always true, regardless of the values that any variables involved might take on; similar to laws or properties. Many of these can be analogous to normal multiplication and addition, particularly when the symbols {0,1} are used for {FALSE, TRUE}. 
\bigskip 
	
\section{TRUTH TABLE}
The Truth Table for the above identities is as follows:
\begin{enumerate}[label=\textbf{(\Alph*})]
	\item \textbf{$ (P\oplus Q)^{\prime} = (P\odot Q)$} \\
where $Y1=(P\oplus Q)^{\prime}, Y2=(P\odot Q)$\\
\bigskip
\begin{table}[ht!]
	\centering
\begin{tabular}{ |c |c |c |c |} 
\hline 
\newline 
	\textbf{P} & \textbf{Q} & \textbf{Y1} & \textbf{Y2} \\ 
\hline
	0 & 0 &1 &1\\   
	0 & 1 &0 &0\\  
	1 & 0 &0 &0\\  
	1 & 1 &1 &1\\  
\hline 
\end{tabular}
	\caption{}
\end{table}
\bigskip
\bigskip

	\item \textbf{$(P^{\prime}\oplus Q)=(P\odot Q)$}\\
where $Y1=(P^{\prime}\oplus Q), Y2=(P\odot Q)$\\
\bigskip
\begin{table}[ht!]
	\centering
\begin{tabular}{ |c |c |c |c |} 
\hline 
\newline 
	\textbf{P} & \textbf{Q} & \textbf{Y1} & \textbf{Y2}\\
\hline  
	0 & 0 &1 &1\\   
	0 & 1 &0 &0\\ 
	1 & 0 &0 &0\\  
	1 & 1 &1 &1\\
\hline 
\end{tabular}
\caption{}
\end{table}
\bigskip

	\item \textbf{$(P^{\prime}\oplus Q^{\prime}) = (P\oplus Q)$} \\ where $Y1=(P^{\prime}\oplus Q^{\prime}), Y2=(P\oplus Q)$\\
\bigskip
\begin{table}[ht!]
	\centering
\begin{tabular}{ |c |c |c |c |} 
\hline 
\newline 
\textbf{P} & \textbf{Q} & \textbf{Y1} & \textbf{Y2}\\ 
\hline 
	0 & 0 &0 &0\\   
	0 & 1 &1 &1\\  
	1 & 0 &1 &1\\
	1 & 1 &0 &0\\
 \hline 
 \end{tabular}
	\caption{}
\end{table}
\bigskip

	\item \textbf{$(P\oplus P^{\prime})\oplus Q = (P\odot P^{\prime})\odot Q^{\prime}$ }\\
where $Y1=(P\oplus P^{\prime})\oplus Q,Y2=(P\odot
    P^{\prime})\odot Q^{\prime}$\\
\bigskip
\begin{table}[ht!]
	\centering
	\begin{tabular}{ |c |c |c |c |} 
\hline 
\newline 
		\textbf{P} & \textbf{Q} & \textbf{Y1} &  \textbf{Y2}\\ 
\hline 
		0 & 0 &1 &0\\   
 		0 & 1 &0 &1\\  
 		1 & 0 &1 &0\\  
 		1 & 1 &0 &1\\  
 \hline 
 \end{tabular}
	\caption{}
\end{table}
\bigskip

\paragraph{}
	Here, Except \textbf{(D)} identity all other identies are valid according to the mentioned truth tables.
\end{enumerate}
\bigskip

\section{Implementation}
\begin{table}[h]
  \centering
  \caption{connections}
  \begin{tabular}{|c|c|c|}
\hline
Arduino pin & INPUT & OUTPUT\\
\hline
5 & P &\\
\hline
6 & Q &\\
\hline
2 & & C\\
\hline
3 & & R\\
\hline
  \end{tabular}
\end{table}

\section{Procedure}
\paragraph{}

   1. Connect the circuit as per the above table.\\
   2. Connect the Output pins C and R to the LED's.\\
   3. Connect the other end of the LED's to the Ground terminal.\\
   4. Connect inputs to Vcc for logic 1,ground for logic 0.\\
   5. Execute the circuits using the below code.\\
   6. Change the values of P,Q in the code and verify the Truth tables respectively.\\

\section{CODE}
\paragraph{}
	The arduino code can be downloaded from the below link.
\begin{center} 
	\fbox{\parbox{8.5cm}{\url{https://github.com/BagiliSalmaBegam/FWC/tree/main/latex }}} 
\end{center}


\end{document}

